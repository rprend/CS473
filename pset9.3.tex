\documentclass{article}

\usepackage{lipsum}

\begin{document}

\section{Problem Description}

Prove that Steiner Tree problem is NP-complete.

\section{Solution}

We prove that the Steiner Tree problem is NP-complete via a reduction of the
Set Cover problem to the Steiner Tree problem.

Given an instance of Set Cover with universe set $U$ of size $n$, a set $M
\subseteq 2^U$ of size $m$, a target size $k$, and an algorithm $F(T, G, n)$ for
determining if a Steiner Tree with $n$ edges exists in $G$ where $T$ is the
set of terminal nodes, which runs in polynomial time in the size of the input
graph, the Set Cover instance can be solved as follows:

Construct a graph $G$, which has 3 sets of vertices: one set $V_1$ corresponding to
the elements of $U$, one set $V_2$ corresponding to the elements of $M$, and
one final set of one vertex $v_f$.

Place an edge between two vertices $u$ and $v$ exactly when:

$$u \in V_1 \land v \in  V_2 \land u \in v$$
or
$$u \in V_2 \land v = v_f$$

Run the algorithm
$$F(V_1 \cup \{v_f\}, G, k+n)$$ 
and return the result.

Graph construction takes $O(n^2m)$, since in the worst case, for each pair of
vertices considered for edge adding, we do a linear search through the
corresponding subset of $U$, which is size $O(n)$. The graph has $1 + m + n$
vertices, and $O(nm)$ edges. Thus, the whole algorithm will run in polynomial
time.

Because the Set Cover problem is given as an NP-complete problem, and given a
polynomial time algorithm to solve Steiner Tree, we can solve Set Cover in
polynomial time, Steiner Tree is also an NP-complete problem.

\section{Proof of Correctness of Reduction}

Claim 1: given an instance of Set Cover, if there is a solution of size $k$,
there is a solution of $k+n$ edges to the corresponding constructed Steiner
Tree instance.

Over the elements of the subsets in this solution, each element of $U$ will be
included at least once. These live in the graph as $V_U \subseteq V_2$.
Therefore, each vertex in $V_1$ will have a connection to at least one of the
$k$ $V_U$ vertices. A valid Steiner tree is the terminal nodes $V_1$, the
selected subset nodes $V_2$, the final vertex $v_f$, exactly one edge
connected to each of $V_1$ where the other end is in $V_2$, and $(v, v_f)
\forall v \in V_2$. The graph connects all the terminal nodes and has $k+n$
edges and is thus a $k+n$ Steiner tree.

Claim 2: Given an instance of Set Cover, if there is a solution with $k+n$
edges to the corresponding constructed Steiner Tree instance, there is a
solution to the set cover instance of size $k$.

First, note that at least $n$ edges of this solution tree must reside in
between $V_1$ and $V_2$, because each vertex in $V_1$ must be connected to the
rest, and there are no direct edges between the vertices of $V_1$.

If there are more than $n$ edges in this section of the tree, we see that some
edges redundantly cover a $V_1$ vertex (type-1), and the rest connect a $V_2$
vertex to $v_f$ (type-2). For any type-1, I will argue that it can either be
disregarded or swapped out for a type-2. Since our solution is a tree,
removing the edge splits it into two sub-trees. These subtrees can be
reconnected by using one of the unused type-2 edges. There is guaranteed to be
at least one of these because if there weren't, the tree would have had a loop
and not have been a tree in the first place. This procedure can be continued
until the current Steiner tree has $k$ type-2 edges.

At this point, a set cover of size $k$ can be easily shown to exist. Since
there are no redundant type-1 edges, each of the $V_2$ vertices connected to
$V_1$ vertices in the tree must be connected to $v_f$ through the type-2
edges. There are $k$ type-2 edges, so there must be $k$ $V_2$ vertices in the
Steiner tree. Between these, they are guaranteed to include all of the
elements in $U$.

\end{document}
