% ---------
%  Compile with "pdflatex hw0".
% --------
%!TEX TS-program = pdflatex
%!TEX encoding = UTF-8 Unicode

\documentclass[11pt]{article}
\usepackage{macros,handout,graphicx}
\usepackage[utf8]{inputenc}		% Allow some non-ASCII Unicode in source
\usepackage{amsmath}
\usepackage{bbm}

\usepackage{algorithm}
\usepackage{algorithmic}


% =========================================================
%   Define common stuff for solution headers
% =========================================================
\Class{CS/ECE 473}
\Semester{Fall 2020}
\Authors{3}
\AuthorOne{Ryan Prendergast}{ryanp4}
\AuthorTwo{Noah Watson}{nwatson3}
\AuthorThree{Lawson }{laws}
%\Section{}

% =========================================================
\begin{document}

% ---------------------------------------------------------



\AuthorOne{Ryan Prendergast}{ryanp4}
\AuthorTwo{Noah Watson}{nwatson3}

\HomeworkHeader{1}{1}

\subsection *{1.a} Describe an algorithm to find an optimal solution $k_1...k_n$ that runs in time polynomial in $k$ and $n$. 
\begin{solution}
We define the function minFuncSum(i,k) which finds the minimum function sum of $T_i... T_n$ subject to the constraint that the non-negative integers $k_i...k_n$ sums to k with the following recurrence: 

\[
	\text{minFuncSum(i,k)} = \begin{cases}
		T_i(k) & \text{if $i = N$} \\
		min_{j=0}^{k} \{T_i(j) + \text{minFuncSum(i+1, k-j)} & \text{otherwise} \\
	\end{cases}
\]
\\ 
We find the minimum function sum of the original problem by calling minFuncSum(1, k). \\

The above recurrence runs in expornential time. However we see that each iteration minFuncSum(i,k) depends on minFuncSum(i + 1, j) for all possible $j = 0 $ to $n$. We memoize in a 2d, $N*k$ array $A$ with the minimum function sum. We also memoize the value of $0 \leq k_i \leq k$ at which the minimum was found. \\ 

$A[i][j].\text{sum}$ holds the minimum of the functions $T_i...T_n$ subject to the constraint that $k_i...k_n$ sums to j; if $i = N$, then $A[n][j].\text{sum} = T_n(j)$. $A[i][j].\text{k}$ holds the value of $0 \leq k_i \leq k$ at which the minimum lies; if $i = N$, then $A[n][j].\text{k}=j$ \\

Due to the dependency ordering, we can define an iterative algorithm to fill in this $N*k$ array. We iterate from right to left, from $i=n$ to $i=1$, filling in the whole column of all possible values of $j$ from $0...k$ at each iteration. \\ 

To get the sequence $k_1...k_n$, we start at $A[1][k]$ after filling in the whole array with the above algo. $A[i][k].\text{k}$ is $k_1$, then we go to $A[2][k-k_1]$. $A[2][k-k_1].\text{k}$ is $k_2$. In general, $k_i= A[i][k-\sum_{j=1}^{i-1} k_j]$. \\ 

See Algorithm 1 bellow. \\

\textbf{Time Complexity}
As seen in Algorithm 1 bellow, filling in the array requires a nested loop over n and k, and each iteration requires $O(k)$ steps to find the minimum. So, the total runtime is $O(k^2*N)$. \\
\textbf{Space Complexity}
The space required is dominated by the array A, which is size $O(N*k)$.
\begin{algorithm}
\caption{Iterative Min Function Sum Series}
\begin{algorithmic}
\FOR{$i = n$ to $i = 1$}
\FOR{$j=0$ to $j=k$}
\IF{$i = n$}
\STATE $A[i][j].\text{sum} = F_i(j)$
\STATE $A[i][j].\text{k} = j$
\ELSE
\STATE $A[i][j].\text{sum} = \text{min}_{h=0}^{h=j} \{ F_i(h) + A[i+1][j-h]$
\STATE $A[i][j].\text{k} = \text{argmin}_{h=0}^{h=j} \{ F_i(h) + A[i+1][j-h]$
\ENDIF
\ENDFOR
\ENDFOR
\STATE $j \gets k$
\FOR{$i=1$ to $i = n$}
\STATE $k_i \gets A[i][j].k$
\STATE $j \gets j - k_i$
\ENDFOR
\end{algorithmic}
\end{algorithm} 
\end{solution}
\end{document}
