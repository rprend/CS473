\documentclass{article}

\usepackage{lipsum}

\begin{document}

\section*{1.a}

We are given a graph $G = (V,E)$, and a vertex $v \in V$, and we want to
construct a graph $G'$ such that $G'$ has the same vertices as $G$ except with
$v$ removed, and the shortest path distances between all remaining vertex
pairs are the same.

If we consider removing $v$, for each individual path of length 2 going
through $v$, we replace it with a single edge of the same total length, thus
bypassing $v$ with the same cost. Then we can remove $v$ from the graph and
return it.

\begin{verbatim}
p1(G = (V,E), v):
    G' := copy(G) without v
    for i in inc(v):
        for j in out(v):
            G'_w(i,j) := min(G'_w(i,j), G_w(i,v) + G_w(v,j))
    return G'
\end{verbatim}

\section*{1.b}

We are given a graph $G$ and a vertex $v$, along with a graph $G'$ generated
from running part a on $G$ and $v$, where we already know all the shortest
path distances in $G'$. We want to compute the remaining shortest path
distances in $G$, which are those from any other vertex to $v$ and $v$ to any
other vertex.

Since we already know the distance from $j$ to $x$ for any $j$ in the
outgoing vertices of $v$, we can compute the minimum distance from $v$ to $x$
by minimizing the sum of that distance with the edge weight $w(v,j)$ over all
choices of $j$.

The same argument applies to computing the minimum distance from $x$ to $v$.
We repeat these steps for all vertices in $G'$.

\begin{verbatim}
p2(G = (V,E), G' = (V-v,E')):
    for x in V-v:
        minval := inf
        for j in G_out(v):
            minval := min(minval, G_w(v,j) + spd(j,x))
        spd(v,x) := minval
    for x in V-v:
        minval := inf
        for j in G_in(v):
            minval := min(minval, G_w(j,v) + spd(x,j))
        spd(x,v) := minval
    return
\end{verbatim}

\section*{1.c}

To obtain an $O(n^3)$ all-pairs-shortest-path algorithm, we can combine the
algorithms from part a and b. The idea is to recursively strip off one vertex
at a time with part a, and on the way back up the recursion tree, to compute
the distances.

First, we choose some vertex $v$ in $G$. we remove it with the part a
algorithm. Since all the remaining shortest path distances remain the same
from part a, when we call our function on $G'$, it will compute the correct
shortest path distances for the subset of vertex pairs excluding $v$.

Then, we use the part b algorithm to determine the distances to and from $v$.
In addition, we calculate the distance from $v$ to itself, using a similar
argument as from part b.

In the base case, there will be only one vertex in the graph. Since the part a
algorithm replaces edges in a simple graph instead of adding multiple edges
between the same vertex pairs, always choosing the shorter edge, there will be
only one edge from our one vertex to itself, and it will be of length
corresponding to the shortest path distance to itself.

\begin{verbatim}
APSP(G = (V,E)):
    v := arbitrary vertex in V
    if |V| = 1:
        spd(v,v) := w(v,v)
        return
    G' := P1(G,v)
    APSP(G')
    P2(G,G',v)
    spd(v,v) := inf
    for x in out(v):
        spd(v,v) := min(spd(v,v), spd(x,v) + w(v,x))
    return 
\end{verbatim}

Runtime:

The first two algorithms are $O(n^2)$. It takes $O(n)$ time to compute the
distance from $v$ to itself. The part c algorithm therefore does $O(n^2)$
work per recursive call, and makes one smaller recursive call. There will be
$n$ recursive calls since each one removes one vertex and the base case is
reached when only one vertex is left. The runtime is $O(n^3)$.

\end{document}

