% ---------
%  Compile with "pdflatex hw0".
% --------
%!TEX TS-program = pdflatex
%!TEX encoding = UTF-8 Unicode

\documentclass[11pt]{article}
\usepackage{macros,handout,graphicx}
\usepackage[utf8]{inputenc}		% Allow some non-ASCII Unicode in source
\usepackage{amsmath}
\usepackage{bbm}

\usepackage{algorithm}
\usepackage{algorithmic}
\usepackage{listings}
\usepackage{xcolor}

\definecolor{codegreen}{rgb}{0,0.6,0}
\definecolor{codegray}{rgb}{0.5,0.5,0.5}
\definecolor{codepurple}{rgb}{0.58,0,0.82}
\definecolor{backcolour}{rgb}{0.95,0.95,0.92}

\lstdefinestyle{mystyle}{
    backgroundcolor=\color{backcolour},
    commentstyle=\color{codegreen},
    keywordstyle=\color{magenta},
    numberstyle=\tiny\color{codegray},
    stringstyle=\color{codepurple},
    basicstyle=\ttfamily\footnotesize,
    breakatwhitespace=false,
    breaklines=true,
    captionpos=b,
    keepspaces=true,
    numbers=left,
    numbersep=5pt,
    showspaces=false,
    showstringspaces=false,
    showtabs=false,
    tabsize=2
}

\lstset{style=mystyle}


% =========================================================
%   Define common stuff for solution headers
% =========================================================
\Class{CS/ECE 473}
\Semester{Fall 2020}
\Authors{3}
\AuthorOne{Ryan Prendergast}{ryanp4}
\AuthorTwo{Noah Watson}{nwatson3}
\AuthorThree{Lawson Probasco}{lawsonp2}
%\Section{}

% =========================================================
\begin{document}

% ---------------------------------------------------------



\HomeworkHeader{4}{2}
\begin{enumerate}[(a)]
\item
\begin{solution}
    \begin{align*}
        \E[X] &= \E[X_1 + ... + X_n] = \Sigma_{i=1}^n \E[X+i] & \text{linearity of expectation} \\
             &= \Sigma_{i=1}^n \Sigma_x xp(1-p)^{i-1} & \text{defn. of E} \\
             &= \Sigma{i=1}^n \frac{1-p}{p} & \text{expectation for geometric} \\
             &= \frac{n(1-p)}{p} & \\
    \end{align*}
    \begin{align*}
        \E[X^2] &= \E[(X_1 + ... + X_n)^2] & X = \Sigma_{i} X_i \\
               &= \E[{X_1}^2 + X_1X_1 + ... + {X_n}^2] & \\
               &= \Sigma_{i=1}^n \E[X_{i}^2] + \Sigma_{i,j \in |n|, i \neq j}^n \E[X_i X_j] & \text{linearity of expectation} \\
        \E[X]^2 &= (\E[X_1] + ... + \E[X_2])^2 & \text{linearity of expectation} \\
               &= \Sigma{i=1}^n \E[X+i]^2 +\Sigma_{i,j \in |n|, i \neq j}^n \E[X_i X_j] & \text{indep. of $X_i$ $X_i$ $i \neq j$} \\
        Var(X) &= \E[X^2]- \E[X]^2 & \text{defn. of var in terms of E} \\
               &= \Sigma{i=1}^n \E[X_i^2] - \Sigma_{i=1}^n \E[X_i]^2 & \text{$\E[X^2]$ and $\E[X]^2$ from above} \\
               &= \Sigma_{i=1}^n (\E[X_i^2] - \E[X_i]^2) & \\
               &= \Sigma_{i=1}^n Var(X_i) & \\
    \end{align*}
\end{solution}

\item
\begin{solution}
    When applying Chebyschev's inequality $Pr[|(X - \E[X])] \geq \epsilon]$ to $Pr[X \geq c\E[X]$, because $X$ is always positive (as is $\E[X]$ naturally), there are two cases for the absolute value. \\
    $X - \E[X]$ where $X >= \E[X] >= 0$ is a positive or zero value, and thus would never deviate from $|X - \E[X]|$. \\
    $X - \E[X]$ where $0 <= X < \E[X]$ is a negative value $v$ in the range $\E[X] <= v < 0$. Thus, in order for Chebyschev's to be applied to $(X - \E[X])$, $\epsilon$ must satisfy $\epsilon \geq -\E[X]$, such that there is no difference of evaluation of $|X - \E[X]| \geq \epsilon$ and $X - \E[X] \geq \epsilon$. Because $\epsilon = c\E[X]$ satisfies this requirement for $c \geq 2$, we get $Pr[X \geq \E[X]] \leq \frac{Var(X)}{\E[X]^2}$ applying Chebyschev's, and using values from the previous part.
\end{solution}

\item
\begin{solution}
    The event $X > t$ is describes the occurrence in which $X > t$ flips are required to yield $n$ flips of heads. As a result, by definition this means that the first $t$ flips yieled fewer than $n$ heads, e.g. $Y < n$. As a result, $Pr[X > t] = Pr[Y < n]$. Considering the case where $Pr[X = t]$, there are $\binom{t-1}{n-1}$ ways to partition the t flips into n groups (dilineated by a successful flip of heads, using stars and bars). Each of these ways to partition have probability $p^n(1-p)^{t-n}$ (based on number of successes and failures), therefore $Pr[X = t] = \binom{t-1}{n-1}p^n(1-p)^{t-n}$. $Pr[Y = n]$ is modeled by a binomial distribution, thus $Pr[Y = n] = \binom{t}{n}p^n(1-p)^{t-n}$ This leads to $Pr[X = t] \leq Pr[Y = n]$, and thus $Pr[X \geq t] \leq Pr[Y \leq n]$.
    \\
\end{solution}

\item
\begin{solution}
    In this application, $\epsilon = c - 1$, e.g. is $\epsilon$ is over the bound $\epsilon \geq 1$. Applying part 1c i for the first applicable portion of this bound, we can say that $Pr[X \geq c\E[X]] \leq e^{-(c-1)^2\E[X]/4}$ for $2 \leq c \leq 2 + \ln{4}$. Similarly for second portion of the bound of $c$,  we can apply part 1c ii and say that $Pr[X \geq c\E[X]] \leq 2^{-(c-1)\E[X]/2}$ for $c \geq \ln{4}$, and thus we have bound $Pr[X \geq c\E[X]]$ for $c \geq 2$.
\end{solution}

\end{enumerate}

\clearpage
\end{document}
