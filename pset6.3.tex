% ---------
%  Compile with "pdflatex hw0".
% --------
%!TEX TS-program = pdflatex
%!TEX encoding = UTF-8 Unicode

\documentclass[11pt]{article}
\usepackage{macros,handout,graphicx}
\usepackage[utf8]{inputenc}		% Allow some non-ASCII Unicode in source
\usepackage{amsmath}
\usepackage{bbm}

\usepackage{algorithm}
\usepackage{algorithmic}
\usepackage{pdfpages}


% =========================================================
%   Define common stuff for solution headers
% =========================================================
\Class{CS/ECE 473}
\Semester{Fall 2020}
\Authors{3}
\AuthorOne{Ryan Prendergast}{ryanp4}
\AuthorTwo{Noah Watson}{nwatson3}
\AuthorThree{Lawson Probasco}{lawsonp2}
%\Section{}

% =========================================================
\begin{document}

% ---------------------------------------------------------



\HomeworkHeader{6}{3}
Let $G = (V, E)$ be a flow network with integer edge capacities. We have seen algorithms that
compute a minimum $s-t$ cut via maximum flow. For the problem below assume that you only
have black box access to an algorithm that given G and nodes s, t outputs a minimum cut
between s and t.

\begin{enumerate}[(a)]
\item
Describe a simple example of a flow-network $G$ and two nodes s, t such that there are two distinct $s-t$ minimum cuts with the same capacity but different number of edges in the cuts. 

\begin{solution}
On the next page, we show two minimum (s,t) cuts of a graph with a different number of edges. We define $s$ to be $0$ and $t$ to be $3$.
\end{solution}

\includepdf[scale=0.6, pages=-]{graph.pdf}


\clearpage
\HomeworkHeader{6}{3}

\item Given G and s, t and an integer k describe an algorithm that checks whether G has a minimum cut with at most k edges.
\begin{solution} We used as inspiration for our solution the modifications described here: \\ https://cs.stackexchange.com/questions/115159/minimum-cut-with-minimum-number-of-edges. 

We break up this problem into three parts. 

\begin{itemize} 
\item First, we modify the flow network G into $G'$ such that the minimum cut in $G'$ is equivalent to the minimum cut in G with the minimum number of edges. We define $G'=(V', E')$ as a flow network with the same vertices as G, $V'=V$, and the same edges $E'=E$. We then modify the edge capacities $C'(e)$ as follows: 
\begin{align*}
C'(e) &= C(e)*(|E|+1) + 1  
\end{align*}
We note that a cut in G of capacity $m$ corrpesponds to a cut in $G'$ with capacity $m*(|E|+1)+l$ where $l$ is the number of edges in the cut. \\ \\
Any min cut in $G'$ will be a min cut in $G$. Assume we have a min cut in $G$ of size $m$, with corrpesponding size in $G'$ of $m*(|E|+1)+l$. If you take any cut capacity in $G$ of size $m+1$ (and all larger capacities must be at least this large since all capacities are integers), the cut capacity in $G'$ will be $(m+1)*(E+1)+l'$. $l'\leq |E|$, so this new capacity will be larger than the cut capacity corrpesonding to the min cut in $G$, $m*(|E|+1)+l$. \\ \\
We note further that any min cut in $G'$ will be a min cut in G with the fewest possible number of cut edges. We defined our capacity function such that the cut capacity in $G'$ increases by 1 for every edge it cuts, thus a cut with fewer edges will have a smaller capacity. \\

\item Second, we use our black box algorithm to find a min cut in $G'$. The cut it provides gives us a minimum cut in $G$ with the minimum number of edges in it. Call this number of edges $l$. \\

\item Finally, we check to see if $G$ has a min cut with at most $k$ edges. If $l\leq k$, then we return True. Otherwise, we return False.

\end{itemize}
\\
\textbf{Analysis} We construct the graph $G'$ in $O(V+E)$ time. We call the runtime of the black box algorithm $BB$, where $BB$ is a $O$ function according to $V$ and $E$. We run the black box algorithm once, and our third step runs in constant time, so we say the whole routine runs in $O(V+E+BB)$ time.

\end{solution}
\end{enumerate}

\end{document}
