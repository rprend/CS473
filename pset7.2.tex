% ---------
%  Compile with "pdflatex hw0".
% --------
%!TEX TS-program = pdflatex
%!TEX encoding = UTF-8 Unicode

\documentclass[11pt]{article}
\usepackage{macros,handout,graphicx}
\usepackage[utf8]{inputenc}		% Allow some non-ASCII Unicode in source
\usepackage{amsmath}
\usepackage{bbm}

\usepackage{algorithm}
\usepackage{algorithmic}
\usepackage{pdfpages}


% =========================================================
%   Define common stuff for solution headers
% =========================================================
\Class{CS/ECE 473}
\Semester{Fall 2020}
\Authors{3}
\AuthorOne{Ryan Prendergast}{ryanp4}
\AuthorTwo{Noah Watson}{nwatson3}
\AuthorThree{Lawson Probasco}{lawsonp2}
%\Section{}

% =========================================================
\begin{document}

% ---------------------------------------------------------



\HomeworkHeader{7}{2}
Let $G = (V, E)$ be an undirected graph. Let $V$ be partitioned into two sets of nodes $R$ and $N$ where $R$ is the set of reliable nodes and N is the set of non-reliable nodes. The reliable nodes never fail but non-reliable nodes and edges can fail. Call $N U E$ the elements. Given two reliable nodes s, t describe an algorithm that computes the minimum number of elements
whose failure results in s and t being disconnected from each other.

\begin{solution}
First, define a flow network $G'=(V', E')$. Include every vertex V in $V'$, and for each undirected edge in $G$, create two directed edges in $G'$, one for each direction. \\ \\
Once this is done, modify $V' \in G'$ such that every non-reliable node $n \in N$ becomes two nodes, here called $n_a$ and $n_b$. Create an edge $(n_a,n_b)$. Redirect all of the incoming edges of $n$ to $n_a$ and create edges for all of the outgoing edges of $n$ which go from $n_b$ to the destination of the corresponding edge. Every edge in $G$ should receive a  capacity of 1 unit. \\ \\
We construct this flow network using brute force. We know that the number of nodes in $G'$ is at most twice that of G (if every node is non-reliable). $G'$ also has every edge E plus one for each additional node (V). This requires $O(V'+E')=O(2V+V+E)=O(3V+E)$ time.   \\ \\
The purpose of this graph modification is to recast the problem as a minimum cut problem. For every edge $v_a \to v_b$ which is cut, it represents removing the vertex $v$ in $N$. For every other edge which is cut, it represents simply removing the edge in $E$. Since all edge weights are 1, removing an edge counts the same as removing a non-reliable vertex. The algorithm will not be able to "remove" reliable edges because they are represented as a single vertex in $G'$. \\ \\
We run a max-flow algorithm on $G'$ (equivalent to a min-cut capacity) to find the solution to the problem. \\

\textbf{Correctness}
We prove that a bijection exists between a number of elements failing in $G$ and a cut capacity in $G'$. Thus, the minimum cut capacity of $G'$ is equivalent to the minimum number of elements whose failure disconnects s and t, and solves the problem. \\ \\
Assume we have a cut in $G'$ with capacity c. Since every edge has a capacity of 1, this means there are c edges cut through. We define a one-to-one mapping f from each of the c edges in G' to an element failing in G as follows. For each of the c edges $e \in E'$, if $e \in E$, we map to the edge in G. If $e \not{\in} E'$, then $e$ must be one of the edges we added in $G'$. These edges all have the form $v_a -> v_b$, so we map to the node $v \in V$. \\ \\
Now, assume we have a collection of c elements in G. We define a one-to-one mapping from each element to an edge in a cut of G'. If an element $x$ is a node $v$, then we know $G'$ has nodes $v_a$ and $v_b$ with exactly one edge between. So, we put $v_a$ in S and $v_b$ in T, increasing the cut capacity by exactly one. If the element $x$ is an edge $v -> u$, we put $v$ in $S$ and $u$ in $T$ in the s,t cut of $G'$, cutting through the edge and adding one to the capacity. \\ \\

\textbf{Time Complexity}
Graph size: The new graph will have $O(|V|)$ vertices because each vertex is turned into at most 2 vertices. The graph will have $O(|N|+|E|)$ edges, where $N$ is $O(V)$. Summed together, the graph construction takes $O(2V+E)$ time. \\ \\
Graph algorithm: Maximum flows can be computed in $O(VE)$ time. For our flow network, this translates to $O(V(N+E))=O(V(V+E))$ time.

\end{solution}


\clearpage
\HomeworkHeader{6}{3}

\item Given G and s, t and an integer k describe an algorithm that checks whether G has a minimum cut with at most k edges.
\begin{solution} We used as inspiration for our solution the modifications described here: \\ https://cs.stackexchange.com/questions/115159/minimum-cut-with-minimum-number-of-edges. 

We break up this problem into three parts. 

\begin{itemize} 
\item First, we modify the flow network G into $G'$ such that the minimum cut in $G'$ is equivalent to the minimum cut in G with the minimum number of edges. We define $G'=(V', E')$ as a flow network with the same vertices as G, $V'=V$, and the same edges $E'=E$. We then modify the edge capacities $C'(e)$ as follows: 
\begin{align*}
C'(e) &= C(e)*(|E|+1) + 1  
\end{align*}
We note that a cut in G of capacity $m$ corrpesponds to a cut in $G'$ with capacity $m*(|E|+1)+l$ where $l$ is the number of edges in the cut. \\ \\
Any min cut in $G'$ will be a min cut in $G$. Assume we have a min cut in $G$ of size $m$, with corrpesponding size in $G'$ of $m*(|E|+1)+l$. If you take any cut capacity in $G$ of size $m+1$ (and all larger capacities must be at least this large since all capacities are integers), the cut capacity in $G'$ will be $(m+1)*(E+1)+l'$. $l'\leq |E|$, so this new capacity will be larger than the cut capacity corrpesonding to the min cut in $G$, $m*(|E|+1)+l$. \\ \\
We note further that any min cut in $G'$ will be a min cut in G with the fewest possible number of cut edges. We defined our capacity function such that the cut capacity in $G'$ increases by 1 for every edge it cuts, thus a cut with fewer edges will have a smaller capacity. \\

\item Second, we use our black box algorithm to find a min cut in $G'$. The cut it provides gives us a minimum cut in $G$ with the minimum number of edges in it. Call this number of edges $l$. \\

\item Finally, we check to see if $G$ has a min cut with at most $k$ edges. If $l\leq k$, then we return True. Otherwise, we return False.

\end{itemize}
\\
\textbf{Analysis} We construct the graph $G'$ in $O(V+E)$ time. We call the runtime of the black box algorithm $BB$, where $BB$ is a $O$ function according to $V$ and $E$. We run the black box algorithm once, and our third step runs in constant time, so we say the whole routine runs in $O(V+E+BB)$ time.

\end{solution}

\end{document}



