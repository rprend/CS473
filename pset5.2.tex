% ---------
%  Compile with "pdflatex hw0".
% --------
%!TEX TS-program = pdflatex
%!TEX encoding = UTF-8 Unicode

\documentclass[11pt]{article}
\usepackage{macros,handout,graphicx}
\usepackage[utf8]{inputenc}		% Allow some non-ASCII Unicode in source
\usepackage{amsmath}
\usepackage{esvect}
\usepackage{bbm}

\usepackage{algorithm}
\usepackage{algorithmic}
\usepackage{listings}
\usepackage{xcolor}

\definecolor{codegreen}{rgb}{0,0.6,0}
\definecolor{codegray}{rgb}{0.5,0.5,0.5}
\definecolor{codepurple}{rgb}{0.58,0,0.82}
\definecolor{backcolour}{rgb}{0.95,0.95,0.92}

\lstdefinestyle{mystyle}{
    backgroundcolor=\color{backcolour},
    commentstyle=\color{codegreen},
    keywordstyle=\color{magenta},
    numberstyle=\tiny\color{codegray},
    stringstyle=\color{codepurple},
    basicstyle=\ttfamily\footnotesize,
    breakatwhitespace=false,
    breaklines=true,
    captionpos=b,
    keepspaces=true,
    numbers=left,
    numbersep=5pt,
    showspaces=false,
    showstringspaces=false,
    showtabs=false,
    tabsize=2
}

\lstset{style=mystyle}


% =========================================================
%   Define common stuff for solution headers
% =========================================================
\Class{CS/ECE 473}
\Semester{Fall 2020}
\Authors{3}
\AuthorOne{Ryan Prendergast}{ryanp4}
\AuthorTwo{Noah Watson}{nwatson3}
\AuthorThree{Lawson Probasco}{lawsonp2}
%\Section{}

% =========================================================
\begin{document}

% ---------------------------------------------------------



\HomeworkHeader{5}{2}
\begin{enumerate}[(a)]
\item
\begin{solution}
    \begin{align}
        Pr_{h \in \mathcal{M}}[h(x) = h(y)] &= Pr[\bigoplus_{i: x_i = 1} M_i = \bigoplus_{i: y_i = 1} M_i] \\
        &= Pr[(\bigoplus_{i: x_i = y_i = 1} M_i) \oplus (\bigoplus_{i: y_i = 0, x_i = 1} M_i) = (\bigoplus_{i: x_i = y_i = 1} M_i]) \oplus (\bigoplus_{i: x_i = 0, y_i = 1} M_i)] \\
        &= Pr[(\bigoplus_{i: y_i = 0, x_i = 1} M_i) = (\bigoplus_{i: x_i = 0, y_i = 1} M_i)] \\
        &= Pr[(\bigoplus_{i: y_i = 0, x_i = 1} M_i) \oplus (\bigoplus_{i: x_i = 0, y_i = 1} M_i) = 0] \\
        &= Pr[h(z) = 0] \\
        &\leq \frac{1}{2^l} \\
        &\leq \frac{1}{m}
    \end{align}
    Annotating the equations down here to avoid clutter. Essentially, the rewriting in (1) is simply using the definition of $h_M$. (2) uses the fact that each pair of distinct input vectors $x,y$ have at least one differing bit, meaning they produce some zero or more shared columns of $M$ and some one or more different columns of $M$. The shared columns on each side are dropped in $(3)$, and the right side is moved over in $(4)$. The resulting equation in $(4)$ is the xor of a set of at least one unique column(s) of $M$, which in turn can be rewritten as some $h(z)$ since this is exactly what the hash function produces anyways. This probability can be evaluated in the following manner - fix one non-zero column $M_{j}$ representing by a corresponding $1$-bit in $z$. This means that $M_{j} = \bigoplus_{i: x_1 = 1, i \neq j} M_i$. This probability is bounded by the $l$ random bits in $M_{j}$, meaning that $Pr[h(z) = 0] \leq \frac{1}{2^l} = \frac{1}{m}$.
\end{solution}

\item
\begin{solution}
    In order to be uniform, $Pr_{h \in \mathcal{M}}[h(x) = i] = \frac{1}{m} \text{ for all } x \text{ and } i$. However, in the case where $x$ is $\vv{0}$, and $i$ is $\vv{1}$, $Pr_{h \in \mathcal{M}}[h(x) = i] = 0$.
\end{solution}

\item
\begin{solution}
    In order to be $2$-uniform (as above), $Pr_{h \in \mathcal{M^+}}[h(x) = i] = \frac{1}{m} \text{ for all } x \text{ and } i$. Considering that there will be (without loss of generality, since independent $h(x),h(y)$ can be flipped) at least one bit in $y$ with a value of $0$, we come to the conclusion that there will always be some column in $M$ - refer to it as $M_i$, in which $M_i$ does not contribute in any way to the calculation of $v_2$ but does contribute to the calculation of $v_1$ as in $Pr[h(x) = v_1 | h(y) = v_2]$. As a result, we can fix this column $M_i$ s.t. $v_2 \oplus M_i = v_1$, e.g. $M_i = v1 \oplus v_2$, more precisely the xor of the unique columns of $v_1$ and $v_2$. Since there are $l$-bits in this vector, the likelihood is then $\frac{1}{2^l} = \frac{1}{m}$. Thus $Pr[h(x) = a \text{ and } h(y) = b] = Pr[h(x) = a] * Pr[h(x) = b] = \frac{1}{m^2}$.
\end{solution}

\item
\begin{solution}

\end{solution}
        In order to be $4$-uniform, $Pr_{h \in \mathcal{M}^+}[\bigwedge_{j=1}^{4} h(x_j) = i_j] = \frac{1}{m^4} \text{ for all distinct } x_1,...,x_4 \text{ and } i_1,...,i_4$. However, consider the case where $x_2$ is the input vector with each even bit set, $x_3$ is the input vector with each odd bit set, and $x_4$ is the input vector $\vv{0}$. In this case then, $Mx_4 = \vv{0}$, meaning $i_4 = b$. If $x_1$ is the input vector $\vv{1}$, this would mean that $i_1 = i_2 \oplus i_3 \oplus i_4$, and as such, these values are not independent and $Pr_{h \in \mathcal{M}^+}[\bigwedge_{j=1}^{4} h(x_j) = i_j] \neq \frac{1}{m^4}$.
\end{enumerate}

\clearpage
\end{document}
