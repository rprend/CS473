% ---------
%  Compile with "pdflatex hw0".
% --------
%!TEX TS-program = pdflatex
%!TEX encoding = UTF-8 Unicode

\documentclass[11pt]{article}
\usepackage{macros,handout,graphicx}
\usepackage[utf8]{inputenc}		% Allow some non-ASCII Unicode in source
\usepackage{amsmath}
\usepackage{bbm}

\usepackage{algorithm}
\usepackage{algorithmic}


% =========================================================
%   Define common stuff for solution headers
% =========================================================
\Class{CS/ECE 473}
\Semester{Fall 2020}
\Authors{3}
\AuthorOne{Ryan Prendergast}{ryanp4}
\AuthorTwo{Noah Watson}{nwatson3}
\AuthorThree{Lawson Probasco}{lawsonp2}
%\Section{}

\usepackage{amsmath}

\begin{document}

\HomeworkHeader{2}{2}

Find the weight of the minimum weight dominating set of size at least $k$, in a tree.

\begin{solution}
First, we define an algorithm to find the mininum weight dominating set of
size exactly $k$.

Define relations:

$$\textsc{OPT}_0(r, k) = \min 
\begin{cases} 
    A + w(r) \\
    \min_{v \in N(r)} B
\end{cases}$$
$$\textsc{OPT}_1(r, k) = C + w(r)$$
$$\textsc{OPT}_2(r, k) = \min
\begin{cases}
    D + w(r) \\
    E
\end{cases}$$

$$n = |N(r)|$$

$$A = \min_{k_1,...k_n} \left[\sum_{x = 1}^n \textsc{OPT}_2(N(r)_x, k_x) \right], \sum_{i=1}^n
k_i = k - 1$$
$$B = \min_{k_1,...k_n} \left[\textsc{OPT}_1(v,k_1) + \sum_{x = 2}^n
\textsc{OPT}_0((N(r) \setminus \{v\})_x, k_x) \right], \sum_{i=1}^n
k_i = k$$
$$C = \min_{k_1,...,k_n} \left[ \sum_{x=1}^n \textsc{OPT}_2(N(r)_x,k_x)\right], \sum_{i=1}^n
k_i = k - 1$$
$$D = \min_{k_1,...,k_n} \left[\sum_{x=1}^n \textsc{OPT}_2(N(r)_x, k_x)\right], \sum_{i=1}^n
k_i = k - 1 $$
$$E = \min_{k_1,...,k_n} \left[\sum_{x=1}^n \textsc{OPT}_0(N(r)_x, k_x)\right], \sum_{i=1}^n
k_i = k$$

The \textsc{OPT} relations are the same, conceptually, as the ones defined in
the lecture slides on the algorithm to find minimum dominating set in a tree.

$\textsc{OPT}_0(r, k)$ is the minimum dominating set of size $k$ in the
subtree rooted at $r$.

$\textsc{OPT}_1(r, k)$ is the minimum dominating set which includes $r$ of size $k$ in the
subtree rooted at $r$.

$\textsc{OPT}_2(r, k)$ is the minimum weight set of size $k$ in the subtree
rooted at $r$ which dominates all nodes in the set of nodes not including $r$
itself.

$A$ through $E$ are the updated, more general forms of the sums of the optimal
solutions for the child nodes. Instead of merely taking the sum of the lowest
weight solutions for the child nodes, we must choose between many different
ways to split the remaining number of selectable nodes between the different
subtrees. This is done by selecting the set $k_1,...,k_n$ that minimizes the
value of the sum of weights, subject to the constraint that the sum of these $k_i$s must be $k-1$ or
$k$, depending on whether the root node is included in the particular case.

Note that the values of these expressions $A$ to $E$ can be computed by the
algorithm from problem 1. Input the target sum of the $k_i$s as the target sum
$k$ from problem 1. For the set of $T$ functions, there will be a $T$ for each
neighboring vertex and for a neighboring vertex $v$, $T_v(x)$ will be
$\textsc{OPT}_t(v, x)$, with $t$ being chosen depending on which
$\textsc{OPT}$ is being used for $v$ in one of $A$ to $E$.

To implement this as a dynamic programming solution, we have to examine the
dependency graph. When looking at the vertex parameter, since any vertex only
depends on \textsc{OPT} values of its children, we can compute the
sets of values for the vertices in post-order, which ensures that any vertex
will have the \textsc{OPT} values of its children available when it is being
computed.

For the $k$ parameter, since for any $\textsc{OPT}_a(v,b)$ is never dependent
on $\textsc{OPT}_c(v,d)$ for any $(a,b,c,d)$, but only on values less than
$v$, the entries of a row corresponding to any vertex can be computed in any
order. 

%TODO base cases
The base cases for the recurrence are the leaves of the tree. For all leaves
$v$,
$$\textsc{OPT}_0(v,0) = \textsc{OPT}_1(v,0) = \infty$$
$$\textsc{OPT}_0(v,1) = \textsc{OPT}_1(v,1) = w(v)$$
$$\textsc{OPT}_0(v,x > 1) = \textsc{OPT}_1(v, x > 1) = \infty$$
$$\textsc{OPT}_2(v,0) = 0$$
$$\textsc{OPT}_2(v,1) = w(v)$$
$$\textsc{OPT}_2(v, x > 1) = \infty$$

%TODO final answer location

The solution to the minimum dominating set of size exactly $k$ is
$\textsc{OPT}_0(r,k)$, where $r$ is the root node.

\textbf{Time complexity analysis:}

First, we compute a post-order traversal order for the tree, which takes
$O(n)$ time. 

Next, we fill out 3 $n$ by $k$ tables, where $n = |V|$. For each set of 3
corresponding entries in the three \textsc{OPT} tables, we perform the
algorithm from problem 1. For A, C, D, and E, this is done once per vertex. For B, this is done $|N(v)|$
times per vertex $v$. However, there are only $n-1$ total neighbors, so the
total number of times $B$ computed is also only $O(nk)$.

The problem 1 algorithm runs in $O(xy^2)$ time, where $x$ is the size number
of $T$ functions and $y$ is the target sum. for any time we run the algorithm,
the input size is $x = N(v), y = k_0$, where $N(v) \in O(n)$ and $k_0 \leq k$,
so any instance of the problem 1 algorithm used here will be $O(nk^2)$ and
therefore the overall runtime for this section is $O(nk*(nk^2)) = O(n^2k^3)$.

We then repeat the entire algorithm for target sum values of
$k$ up to $n$, or $n - k$ times, to find the minimum weight dominating set of size at least k.
This yields a total runtime of $O(n^2k^3(n-k))$.

\end{solution}

\end{document}
