% ---------
%  Compile with "pdflatex hw0".
% --------
%!TEX TS-program = pdflatex
%!TEX encoding = UTF-8 Unicode

\documentclass[11pt]{article}
\usepackage{macros,handout,graphicx}
\usepackage[utf8]{inputenc}		% Allow some non-ASCII Unicode in source
\usepackage{amsmath}
\usepackage{bbm}

\usepackage{algorithm}
\usepackage{algorithmic}


% =========================================================
%   Define common stuff for solution headers
% =========================================================
\Class{CS/ECE 473}
\Semester{Fall 2020}
\Authors{3}
\AuthorOne{Ryan Prendergast}{ryanp4}
\AuthorTwo{Noah Watson}{nwatson3}
\AuthorThree{Lawson Probasco}{lawsonp2}
%\Section{}

% =========================================================
\begin{document}

% ---------------------------------------------------------



\HomeworkHeader{3}{3}
\subsection *{3.a}
Write down a description of randomized quick selection in pseudocode. Show that the expected depth of the recursion of randomized quick selection is $O(log n)$, and that the
expected running time is $O(n)$.

\begin{solution}

We provide the following pseudocode for randomized quickselect on an array A of size n. It takes the arguments l,j as indeces into A: $l \leq n$, $r \leq n$ and $r >= l$. We are looking to find the kth largest element in A. We name the function RQS. We use the standard partition subroutine which, given an array
\begin{algorithm}
\caption{RandomizedQuickSelect(l, r, k)}
\begin{algorithmic}
\IF {$l == r$}
\STATE $return$
\ENDIF
\STATE $pivot \gets RandInt(l..r)$
\STATE $\text{partition}(l,r,pivot)$
\IF {$pivot == k$}
\STATE $\text{return } A[k]$
\ELIF {$ $ }
\ELSE 
\STATE $\text{return } RQS(pivot+1, r, k)$
\ENDIF
\end{algorithmic}
\end{algorithm} 

\end{solution}
  \\
We first define the recurrence $\bar{T}(n)$ to be the expected value of the amount of work on a problem of size n. We know that the partition function requires $n-1$ comparisons to solve a problem of size n.
\begin{align*}
\bar{T}(n) &= n - 1 + max[\bar{T}(k-1), \bar{T}(n-k)]
\end{align*}

We define a good pivot to be in the middle 50\% of elements and a bad one to be in the lower 25 or upper 75.  In the event of a good pivot, 
\begin{align*}
\bar{T}(n) &\leq n - 1 + \bar{T}(3/4 * n)
\end{align*}
and in the event of a bad pivot
\begin{align*}
\bar{T}(n) &\leq n - 1 + \bar{T}(n)
\end{align*}
Because the pivot is selected uniformly randomly, there is equal probability of $1/2$ of both. So, 
\begin{align*}
\bar{T}(n) &\leq 2n - 2 + .5*\bar{T}(n) + .5*\bar{T}(3/4 * n) \\
\bar{T}(n) &\leq 4n - 4 + \bar{T}(3/4 * n)
\end{align*}
\\ By the master theorem, we know that the right side recurrence runs in $O(n)$ time, and this serves as an upper bound.
\\

Next we define the recurrence $\bar{D}(n)$ to be the expected value of the depth of the recursion tree of a problem of size n. We know that each call to the function adds one to the recursion tree depth.
\begin{align*}
\bar{D}(n) &= 1 + max[\bar{D}(k-1), \bar{D}(n-k)]
\end{align*}
We will use the same definition of a good and bad pivot as previously: define a good pivot to be in the middle 50\% of elements and a bad one to be in the lower 25 or upper 75.  In the event of a good pivot, 
\begin{align*}
\bar{D}(n) &\leq 1 + \bar{D}(3/4 * n)
\end{align*}
and in the event of a bad pivot
\begin{align*}
\bar{D}(n) &\leq 1 + \bar{D}(n)
\end{align*}
Because the pivot is selected uniformly randomly, there is equal probability of $1/2$ of both. So, 
\begin{align*}
\bar{D}(n) &\leq 2 + .5*\bar{D}(n) + .5*\bar{D}(3/4 * n) \\
\bar{D}(n) &\leq 4 + \bar{D}(3/4 * n)
\end{align*}
\\ By the master theorem, we know that the right side recurrence runs in $O(logn)$ time, and this serves as an upper bound.

\end{solution}

\clearpage


\HomeworkHeader{3}{3}

\subsection *{3.b} 

\begin{solution} 

\end{solution}

\end{document}
