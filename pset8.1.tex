\documentclass{article}

\begin{document}

\section{Problem description}

We have access to a minimum-cost $(s,t)$-flow algorithm. Reduce the shortest
$(s,t)$-path problem to the minimum-cost $(s,t)$ flow problem.

\section{Solution}

Our input to the $(s,t)$ flow problem is a graph $G = (V,E)$ with edge
lengths $l(e), e \in E$, and start and end vertices $s$ and $t$.

The reduction is as follows:

Define a flow network $G' = (V', E')$. 

Each edge and vertex in $G$ will have a corresponding edge or vertex in $G'$, where
every edge capacity range is $[0, \infty]$ and the cost $\$(e)$ of each edge $e$ in
$G'$ is the same as $l(e)$ in $G$. The balance function of $G'$ will be $-1$ for
$s$, $1$ for $t$, and 0 for all other vertices. $G'$ can be constructed in
$O(|V| + |E|)$ time by brute force. The shortest path length in
$G$ will correspond exactly to the minimum-cost flow in $G'$, or "Negative
Cycle" if the minimum-cost flow returns the same.

\section{Proof of correctness}

Negative cycles: A negative cycle in $G$ implies no feasible minimum-cost
$(s,t)$ flow in $G'$, because an arbitrarily small-cost flow can be created by
first finding some $(s,t)$ flow and then adding an arbitrary amount of flow
cycling through the corresponding negative-cost cycle in $G'$.

Note that this doesn't work if we are only trying to fail and detect negative
cycles in $G$ when the cycle is along a walk from $s$ to $t$, since flow can
be added to a cycle anywhere in the graph while all vertices remain balanced.

No negative cycles:

Let $f$ be a minimum-cost $(s,t)$ path, with cost $k_f$, and
$p$ be a shortest $(s,t)$ path, with length $k_p$.

To prove $k_f = k_p$, we first need to prove that a minimum cost $(s,t)$ flow
in $G'$ will never have an edge with flow value not equal to 1 or 0, provided there
are no negative-cost cycles.

If $(u,v)$ is a flow edge in $f$, where the flow value of $(u,v)$ is greater than 1,
there exists a path from $s$ to $u$ and from $v$ to $t$. This is true because
there are no negative cycles in the graph, and therefore if $(u,v)$ was not
along a path from $s$ to $t$, it must be in a cycle due to the balance
constraints, and a smaller cost flow could be found by removing that cycle.

Since these two paths exist, choose the two such paths such that minimize the
total cost of the paths $s \to u$ and $v \to t$. We can then create an $(s,t)$
flow of cost less than $k_f$ by setting all edges along the path $s \to u \to
v \to t$ to flow value 1. This new flow is a minimum cost flow of those containing
$(u,v)$. We know it is cheaper than $f$ because the flow value of $(u,v)$ is
less than in $f$, and the total cost of the other edges is at least as small
as the total cost of the other edges in $f$, because of the balance
constraint, and the fact that it was selected using the cheapest paths. This is a
contradiction which implies that no such edge can exist in $f$.

If $(u,v)$ is a flow edge in $f$, where the flow value of $(u,v)$ is in range
$(0,1)$, there must exist at least 2 flow paths from $s$ to $t$ in $f$, or in other
words, some quantity of the 1 unit of outflow from $s$ is traveling to $t$ by
a different path than some other quantity. Clearly, however, one of these
routes is cheaper, and it would be cheaper to send all flow through it, which
is feasible because all the upper bound capacities are $\infty$. This is a
contradiction which implies no such edge can exist. Therefore all flow values
must be either 1 or 0.

Case 1: $k_f < k_p$

In this case we can construct a shorter length $(s,t)$ path. Simply take the
non-zero flow edges in $f$ and construct a path out of them. They are
guaranteed to be a path because of balance rules plus the 0/1 requirement
we already proved, plus the non-use of cycles in $f$ also previously proved.
Because of the exact cost/length correspondence, the cost of this path is
$k_f$, which is a contradiction.

Case 2: $k_f > k_p$

In this case we can construct a smaller-cost $(s,t)$ flow by setting the flow
value of all edges in $p$ to 1, and all the others to 0. It is easy to see
this is a valid flow with cost $k_p$, which contradicts $f$ being the cheapest
flow. Therefore $k_f = k_p$. 

\end{document}
