\documentclass{article}

\begin{document}

\section{}

Prove that in any Eulerian graph, if there are $k$ edge-disjoint paths from $a$
to $b$, there are $k$ edge-disjoint paths from $b$ to $a$:

First define a flow network $G$ in terms of the input graph. Assign a capacity
of 1 to each edge in $G$. This graph has some maximum flow of capacity $m$.
Using the argument from page 353 of the Algorithms Textbook, $m$ will also be the
maximum number of edge-disjoint paths from $a$ to $b$, and $m \geq k$.

By the maxflow mincut theorem, the minimum $(a,b)$ cut of the graph is of capacity
$m$.

Since this cut is a partitioning into 2 sections of an Eulerian graph, the
number of edges from $L$ to $R$, or $m$, is equal to the number of edges from $R$ to
$L$. This same partitioning is also a $(b,a)$ cut with capacity $m$. It is the
smallest $(b,a)$ cut because if there was a smaller one, it would have
capacity less than $m$, meaning it would have less than $m$ incoming or
outgoing edges, which would imply there was a smaller $(a,b)$ cut than the one
we found.

Then, again by the maxflow-mincut theorem, the maximum $(b,a)$ flow is $m$,
and by the textbook argument, there exists a set of $m$ edge-disjoint paths in
the input graph from $b$ to $a$. Since $m \geq k$, there is also a set of $k$
edge-disjoint paths from $b$ to $a$.

\section{}

Proof that any $d$-regular bipartite graph has a perfect matching.

Define a flow network $G$ based on the graph, by adding a vertex $a$ and $b$,
connecting $a$ to all vertices in $L$, $b$ to all vertices in $R$, and setting
all edge capacities to 1.

Note that in a d-regular bipartite graph, there are $\frac{|V|}{2} = k$
vertices in each of $L$ and $R$. If there were more vertices on one side,
there would be too many outgoing edges to cover the fewer vertices on the
other side, assuming $d > 0$.

Let $m$ be defined as the maximum $(a,b)$ flow of $G$.

By the maxflow-mincut theorem, $m$ is also the size of the minimum $(a,b)$ cut.

I claim that the smallest cut of the graph is of size $k$.

Any other cut has at least one vertex besides $a$ and $b$ in each of the
partition sets.

Let set $L_a$ be the set of vertices in the left side of the bipartite graph
which are in a partition set with $a$, and similarly for $L_b, R_a, R_b$.

if $|L_a| \geq |R_a|$, there will be at least $d(|L_a| - |R_a|)$ edges from $L_a$
to $R_b$.

We also know that $|R_b| \geq |L_a|$, since $|L_a|+|L_b| =|R_a|+|R_b|=k$.

We want to find the minimum number of edges from $R_a$ to $L_b$. We know that
there are at least $d(|L_a| - |R_a|)$ edges from $L_a$ to $R_b$. This leaves
$d(|R_b| - |L_a| + |R_a|)$ connections left that $R_b$ must fill. We then get
$d(|L_b| - |R_b| + |L_a| - |R_a|)$ edges between $R_a$ and $L_b$. There will
then be a total of $d(|L_a|-|R_a|) + d(|L_b| - |R_b| + |L_a| - |R_a|) + |R_a|
+ |L_b|$ edges.

Because of the relationship $|L_a|+|L_b| =|R_a|+|R_b|=k$, this can be
simplified to $(d-1)(|L_a| -|R_a|) + k$. This will always be at least $k$
because we assumed $|L_a| \geq |R_a|$, if $d \geq 1$.

The same argument applies for $|L_a| \leq |R_a|$ to show that the minimum
possible cut is $k$.

Since $k$ is the minimum cut capicity, it is also the maximum flow capacity,
and this flow can be separated into $k$ edge-disjoint paths. These
edge-disjoint paths correspond to a matching of size $k$ in the original
graph. Each left vertex has only one edge going to it, and each right vertex
has only one edge coming from it in these paths, no vertices are shared
between edges that were in the original graph. There are $k = \frac{|V|}{2}$
edges aside from the extra edges we added, so it is a perfect matching.

\end{document}
