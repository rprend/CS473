\documentclass{article}

\begin{document}

\section{}

Prove that in any Eulerian graph, if there are $k$ edge-disjoint paths from $a$
to $b$, there are $k$ edge-disjoint paths from $b$ to $a$:

First define a flow network $G$ in terms of the input graph. Assign a capacity
of 1 to each edge in $G$. This graph has some maximum flow of capacity $m$.
Using the argument from page 353 of the Algorithms Textbook, $m$ will also be the
maximum number of edge-disjoint paths from $a$ to $b$, and $m \geq k$.

By the maxflow mincut theorem, the minimum $(a,b)$ cut of the graph is of capacity
$m$.

Since this cut is a partitioning into 2 sections of an Eulerian graph, the
number of edges from $L$ to $R$, or $m$, is equal to the number of edges from $R$ to
$L$. This same partitioning is also a $(b,a)$ cut with capacity $m$. It is the
smallest $(b,a)$ cut because if there was a smaller one, it would have
capacity less than $m$, meaning it would have less than $m$ incoming or
outgoing edges, which would imply there was a smaller $(a,b)$ cut than the one
we found.

Then, again by the maxflow-mincut theorem, the maximum $(b,a)$ flow is $m$,
and by the textbook argument, there exists a set of $m$ edge-disjoint paths in
the input graph from $b$ to $a$. Since $m \geq k$, there is also a set of $k$
edge-disjoint paths from $b$ to $a$.

\section{}

Proof that any $d$-regular bipartite graph has a perfect matching.

Define a flow network $G$ based on the graph, by adding a vertex $a$ and $b$,
connecting $a$ to all vertices in $L$, $b$ to all vertices in $R$, and setting
all edge capacities to 1.

Note that in a d-regular bipartite graph, there are $\frac{|V|}{2} = k$
vertices in each of $L$ and $R$. If there were more vertices on one side,
there would be too many outgoing edges to cover the fewer vertices on the
other side, assuming $d > 0$.

Let $m$ be defined as the maximum $(a,b)$ flow of $G$.

By the maxflow-mincut theorem, $m$ is also the size of the minimum $(a,b)$ cut.

I claim that the smallest cut of the graph is of size $k$.

The cut $(\{a\}, G\setminus \{a\})$ is of size $k$.

The cut $(G\setminus \{b\}, \{b\})$ is of size $k$.

Any other cut is size $\geq k$ because it will include at least one of the original
(not $a$ or $b$) vertices in the left partition and in the right partition, as
those are the remaining cases.

For any vertex in the right side of the bipartite graph, it will contribute 1 to the
cut size. If it is in the $a$ side of the cut, the edge neighboring it with
$b$ will be counted. If it is in the $b$ side of the cut, if it is connected
to a vertex in the left side of the bipartite graph which is in the $a$ side
of the cut, the

\end{document}
