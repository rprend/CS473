% ---------
%  Compile with "pdflatex hw0".
% --------
%!TEX TS-program = pdflatex
%!TEX encoding = UTF-8 Unicode

\documentclass[11pt]{article}
\usepackage{macros,handout,graphicx}
\usepackage[utf8]{inputenc}		% Allow some non-ASCII Unicode in source
\usepackage{amsmath}
\usepackage{bbm}

\usepackage{algorithm}
\usepackage{algorithmic}


% =========================================================
%   Define common stuff for solution headers
% =========================================================
\Class{CS/ECE 473}
\Semester{Fall 2020}
\Authors{3}
\AuthorOne{Ryan Prendergast}{ryanp4}
\AuthorTwo{Noah Watson}{nwatson3}
\AuthorThree{Lawson Probasco}{lawsonp2}
%\Section{}

% =========================================================
\begin{document}

% ---------------------------------------------------------



\HomeworkHeader{2}{3}

We define the graph $G=(V,E)$ for all three subproblems as follows: create $n$ vertices, one for each currency, and there are $N^2$ directed edges such that the edge $v_i v_j = -log(Exch[i][j])$ and $v_j v_i = -log(Exch[j][i])$. 

We see that taking the negative log of the exchange rates means we add rather than multiply the edge weights to find the amount of money earned. Let A be the initial amount of currency, $A'$ be the amount after exchanging and $R_1...R_k$ the exchange rates.

\begin{align*}
A' &= A_0 * \prod_{i=1}^{k} R_k \\
-log(A') &= -log ( A_0 * \prod_{i=1}^{k} R_k ) \\
-log(A') &= -log(A_0) + \sum_{i=1}^{k} -log(R_k)
\end{align*}

Each edge is one element of the summation. Note that taking the negative log of an exchange rate >1 will result in a negative number, and <1 will result in a positive number. 
 
\subsection *{3.a} Describe an algorithm that returns an array $MaxAmt[1 .. n]$, where $MaxAmt[i]$ is the maximum amount of currency i that you can obtain  by trading, starting with one unit of currency 1, assuming there are no arbitrage cycles.

\begin{solution}
If we start an amount of currency $A$, and we trade to get more currency (i.e an exchange rate >1), this means we traverse a negative-length edge in our graph G.  Thus, the problem of finding the maximum amount of currency you can obtain starting with one unit of currency i is equivalent to the single source shortest paths problem in G starting at $v_i$. From the equation above, we see that $-log(A')=-log(A)+\sum_{i=1}^{k} -log(R_k)$. The last node in the shortest path is the node at which we have the most currenccy. \\ 

We start the path-finding problem with $-log(A)$ units. Each edge we add the neative log of the exchange rate at that edge. The total . Since we have negative edge lengths, we use the Bellman-Ford algorithm which runs in $O(VE)$ time. There are no arbitrage cycles which means there are no negative length cycles (see part b).  \\

To get the array MaxAmt$[1..n]$, we solve the shortest path problem starting at index 1, then get the distance to each node. See Algorithm 1 bellow. \\

\textbf{Time Complexity}
Constructing the graph takes $O(V+E)=O(N^2)$ time.
Bellman Ford runs in $O(VE)=O(N^3)$ time, which dominates the total runtime of $O(N^3)$. \\

\textbf{Space Complexity} 
Space complexity is dominated by the exchange rates matrix of size $O(N^2)$ used to construct the graph.

\begin{algorithm}
\caption{MaxAmt[1...n]}
\begin{algorithmic}
\STATE $\text{Run Bellman Ford SSSP on G from }v_1 \text{ with initial value of } -log(1).$
\FOR{$i = 1$ to $i = n$}
\STATE $L \gets \text{distance from } v_1 \text{ to } v_n \text{ from the Bellman-Ford above}$
\STATE $MaxAmt[i] = 10^{-L}$
\ENDFOR
\end{algorithmic}
\end{algorithm} 


\end{solution}

\clearpage


\HomeworkHeader{2}{3}

\subsection *{3.b} Describe an algorithm to determine whether the given matrix of currency exchange rates creates an arbitrage cycle. 

\begin{solution} We will costruct the same graph G as used in part (a).\\ 

We recognize that an arbitrage cycle is equivalent to the existence of a negative length cycle in G. We show this as follows: Suppose we start with A units of currency i. An arbitrage cycles is a cycle of exchange rates $R_1...R_k$ ending back at currency i such that $A*\prod_{i=1}^{k} R_i > A$. We must end at the same currency we started with. In our graph G, multiplying by a positive exchange rate is equivalent to adding a negative value. A series of rates $R_1*R_2*...*R_k>1$ that defines an arbitrage cycle, when we take the negative log of it, is a sequence of negative values $-log(R_1)-log(R_2)-...-log(R_k) < 0$, \\

To detect arbitrage cycle, we simply run Bellman Ford on our graph G. If it detects a negative cycle, there is an arbitrage cycle; if it does not detect a negative cycle there is no arbitrage. \\

\textbf{Time Complexity} The runtime is dominated by the Bellman Ford runtime, which is $O(VE)=O(V^3)$.

\textbf{Space Complexity} The space is dominated by the graph, which requires an adjacency matrix of size $O(N^2)$.


\end{solution}


\clearpage


\HomeworkHeader{2}{3}

\subsection *{3.c} Modify your algorithm from part (b) to actually return an arbitrage cycle, if it exists. 

\begin{solution} We will costruct the same graph G as used in part (a).\\ 

We run Bellman Ford on our graph G, as in part (b), to detect an arbitrage cycle. However, we modify the cycle detection process to walk through the cyle and return the nodes in it. See Algorithm 2 bellow: \\

This algorithm modifies the negative cycle detection routine in the Bellman Ford algorithm. Before we run this routine, we must run the $O(VE)$ path distance routine. Since this part is unchanged from the standard Bellman-Ford algorithm, we do not reprint it here. Note that d[v] is the Bellman-Ford distance at note v, and p[v] is the parent node of the node v. Both of these are part of the standard implementation of Bellman-Ford.

\begin{algorithm}
\caption{Bellman Ford Negative Cycle Modification}
\begin{algorithmic}
\STATE $cycle \text{ is a vector of vertices}$
\FOR{each edge e = (u,v) in G}
\IF{$\text{dist[u] + weight(e) < dist[v]}$}
\STATE $\text{ Negative cycle detected containing node v}$
\STATE $curr \gets p[v]$
\WHILE{$curr != v$}
\STATE{$cycle.push(curr)$}
\STATE{$cycle \gets p[curr]$}
\ENDWHILE
\ENDIF
\ENDFOR
\STATE $cycle.reverse()$
\RETURN $cycle$
\end{algorithmic}
\end{algorithm} 


\textbf{Time Complexity} The modified cycle detection routine runs in $O(VE)$ time. This is the same as the runtime for the standard Bellman-Ford routine, so the total runtime is $O(VE)$.

\textbf{Space Complexity} The space is dominated by the graph, which requires an adjacency matrix of size $O(N^2)$.


\end{solution}

\end{document}
